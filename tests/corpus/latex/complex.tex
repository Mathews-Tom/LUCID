\documentclass{article}
\usepackage{amsmath}
\usepackage{graphicx}
\usepackage{amsthm}

\newtheorem{theorem}{Theorem}

\title{A Complex Document}
\author{Test Author}

\begin{document}

\maketitle

\begin{abstract}
This abstract summarizes the key findings of our research into document
parsing and reconstruction algorithms.
\end{abstract}

\section{Introduction}

We present our results in the following sections. Previous work by
\cite{knuth1984} established the foundation. See also \citep{lamport1994}.

\begin{figure}[h]
\centering
\includegraphics[width=0.5\textwidth]{placeholder.png}
\caption{An example figure with a descriptive caption that contains prose.}
\label{fig:example}
\end{figure}

\section{Methods}

\begin{itemize}
\item First item describes the parsing approach.
\item Second item covers the reconstruction algorithm.
\item Third item addresses validation.
\end{itemize}

\begin{enumerate}
\item Step one of the process.
\item Step two involves computation.
\end{enumerate}

\begin{theorem}
For all documents $d$ in the corpus, the round-trip identity holds:
$\text{reconstruct}(\text{parse}(d)) = d$.
\end{theorem}

\begin{proof}
The proof follows from position tracking. Each chunk records its exact
start and end positions in the original string.
\end{proof}

\begin{table}[h]
\centering
\begin{tabular}{|l|r|}
\hline
Format & Files \\
\hline
LaTeX & 5 \\
Markdown & 5 \\
Plain text & 3 \\
\hline
\end{tabular}
\caption{Test corpus composition.}
\label{tab:corpus}
\end{table}

\begin{verbatim}
This is verbatim text that should be preserved exactly.
  Including whitespace and $math$.
\end{verbatim}

\section{Conclusion}

The parser correctly handles all tested document structures.

\bibliography{references}
\bibliographystyle{plain}

\end{document}
